\section{Introduction}

THIS document provides a guide for you to use for your Thesis A report. It is also intended to provide you with some structure and to help you understand the standard of writing required for your final submission – take it as an opportunity to get right into the habit of high-quality technical reporting. The style of writing here, including how figures are labeled, how referencing is done, and the general flow of a research report (that will be fleshed out into your thesis), is what is expected of you. You can use this guide fully or just reproduce your own, but font sizes and margins should be similar to this guide. Your report should have a similar tone and style to the peer-reviewed literature that you have already been reading. It is preferred that you speak in the “scientific voice”, i.e. “the humidity was calculated”, rather than “I calculated the humidity”. Keep the language objective, and be specific where you can be; for instance use “significantly altered from the test conditions” and “20\% greater”, rather than “amazingly, it was drastically different from the test conditions” and “a bit more”. % Percentage symbols are used in Latex to make comments so to use a % symbol in text use the above syntax
	\\ % These are used to create a new line
	\\
In this section, you should introduce your topic (to a reader who is an engineer, but may not be fully familiar with your topic – this is for the benefit of readers other than your supervisor, i.e. the moderator or incoming future students). Here you need only reference very significant literature, and provide an overview of what the field is, what your project is, why your project is important, and very briefly how you intend(ed) on going about the work. It should set up the literature review – the reader now knows roughly what they are reading about and what they are looking for in the coming section(s). 
	\\
	\\ % \textbf{} is used to bold the included words
The introduction might run to \textbf{about 1 to 1.5 pages}, assuming a figure or two showing something highly useful for the reader, such as a picture of the problem (perhaps with annotation), and maybe some kind of flowchart showing the workflow. 
