\section{Research Question and Project Plan}

This section will start with a clear statement on your research question, i.e. what you want to discover in relation to the already available literature and its gaps (connect to previous section). Hypothesis and aims at the basis of your research will also be presented to detail your research question, again in relation to what has been already observed in literature (e.g. a particular aspect is not considered because multiple studies have shown it is not relevant). After detailing your research question, you will describe with technical detail how you are going to conduct your research (research plan). In particular you should discuss:

% Use itemize to create dot points
\begin{itemize}
    \item your proposed solution/experimental methodology to address the research question;
    \item your thesis timeline (possibly with a Gantt chart or some kind of dated mindmap, which can go in appendix, and can be referenced to in this section) and a justification of time allocation for each task;
    \begin{itemize}
        \item Two Gantt charts, see if you are allowed to create them on landscape A4 templates
        \item Add a justification for each time allocation
        \item Create affordable delay periods and explain why you included that as well (Check MECH4100's feedback on your Gantt chart to ensure that it is perfect)
    \end{itemize}
    \item the resources you have identified as available to your research; and
    \begin{itemize}
        \item UNSW Past Research Papers (Sophie, Ishan, Ziao Zhou, Matt's RIR, the Published Report with Svetlana)
        \item Research Papers published by other Universities (Add them here as you find them)
        \item Human resources that are currently working on the project/can help in the Aero Lab (Charitha de Silva, Matthew Deng, Shubneet Sodhi, Mark Zhai)
        \item Books that I may need to check out of the UNSW library (add them here as you go)
        \item Online tools/software (MATLAB, FLIR, Overleaf, Excel, Teams, Outlook, Mendeley)
        \item Physical Locations (Aero Lab, ENG Maker-space, UNSW Library)
    \end{itemize}
    \item the required training and upskilling that you will need to obtain.
    \begin{itemize}
        \item Aero Lab Induction (All Online Training Modules, Physical Induction from Mark Zhai)
        \item Aero Lab Tour (Physical Induction by Matthew Deng)
        \item Maker-space Induction (Workshop Safety Badge (includes Online Module, Quiz and a Physical Induction) - YET TO DO
        \item MATLAB Induction (From Matthew Deng) - COMPLETED BUT THERE IS NO OFFICIAL CERTIFICATE, SO PERHAPS AN ACKNOWLEDGEMENT WILL BE FINE
        \item Online MATLAB Tutorials/Workshops because it is a big part of the project - YET TO DO
        \item Online FLIR tutorials/workshops because it is how the data is processed - YET TO DO
    \end{itemize}
\end{itemize}
Try to corredate textual descriptions with visual aids (e.g. pictures of your experimental rig).
This section is \textbf{3-5 pages} long. 